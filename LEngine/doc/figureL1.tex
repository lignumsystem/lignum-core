\section{L-system mimicking pine growth}\label{sec:L1}
The L system  mimicking pine  growth starts with  one segment  and one
bud.  The main axis, ($A  = 1$),  creates one  segment and  four branches
forking off.  From then on side  branches ($A > 1$) create one segment
and  two additional  branches.  Ramification  stops after  third order
branches. Bending  of the branches  is modeled by pitching  the second
order branches ($A = 2$) down in the $Bend$ module.

%\begin{figure}[p]
%% \begin{picture}(1,1)
%% \put(0,0){\line(1,0){370}}
%% \end{picture}
\begin{verbatim}
open Pine;
const double PI = 3.1415926535897932384;
module F(double);
module B(int,double);
module Pitch(double);
module Roll(double);
module Turn(double);
module Bend(double);

Start:{produce F(0.30)SB()EB()B(1,1.0);}
B(A,l):
{
   if (A==1)
   produce F(l) SB() Pitch(PI/4.0)  B(A+1,l*0.6) EB() 
                SB() Roll(PI/2.0) Pitch(PI/4.0) B(A+1,l*0.6) EB()
                SB() Roll(PI) Pitch(PI/4.0)  B(A+1,l*0.6) EB() 
                SB() Roll(3.0*PI/2.0) Pitch(PI/4.0)  B(A+1,l*0.6) EB()
           B(A,l*0.9);
  else if (A==2)
  produce  Bend(0.3) F(l) SB() Turn(PI/4.0)  B(A+1,l*0.4) EB() 
                          SB() Turn(-PI/4.0) B(A+1,l*0.4) EB()
           Bend(-0.2)B(A,l*0.6);
  else if (A==3)
  produce F(l) SB() Turn(PI/4.0)  B(A+1,l*0.3) EB() 
               SB() Turn(-PI/4.0) B(A+1,l*0.3) EB()
          B(A,l*0.4);
  else
  produce F(l) SB() EB() B(A,l);
}
Bend(s):
{
   produce Pitch(bend);
}
close Pine;

\end{verbatim}
%% \begin{picture}(1,1)
%% \put(0,0){\line(1,0){370}}
%% \end{picture}
%\hline
%\caption{L-system mimicking pine growth}\label{fig:L1}
%\end{figure}