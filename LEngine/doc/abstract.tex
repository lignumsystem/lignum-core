\abstract{
  
  LIGNUM  is  a   functional-structural  tree  model  that  represents
  coniferous and broad-leaved trees  with four modelling units closely
  resembling  the  real  structure  of  trees.   The  units  are  tree
  segments,  the  tree axes,  branching  points  and buds.   Metabolic
  processes are  explicitly related to  the structural units  in which
  they are  taking place.  Here we  enhance the model  LIGNUM with the
  possibility to formally describe  the architectural development of a
  tree with Lindenmayer systems.
  
  We give sample applications of  the system to model single trees and
  forest stand.  Finally we discuss our approach  and its consequences
  for the future development of the model.

\bf Key words\rm: LIGNUM, Lindenmayer systems, functional-structural tree model.  
}
