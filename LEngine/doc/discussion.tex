\section{Discussion}

Trees  are  complex modular  systems  [Smith  00].   Their growth  and
development  has  been studied  intensively  during  the last  century
mainly experimentally.   This has resulted statistical  models and the
emphasis has been  on the details of measurements,  the arrangement of
experiments and  the use and development of  statistical methods [Hari
99]. LIGNUM  is an attempt to  link a detailed  tree architecture with
the  knowledge in tree  physiology based  on measurable  functions and
activities in a tree. The attraction  with such approach is that it is
possible to  make ontological arguments  (i.e. reasoning based  on the
being itself and its existence) of the development of trees [Perttunen
01].  For  example a modeling framework for  different formulations of
flushing of  buds or number of  leaves and their  orientation based on
light regime or resource allocation is readily available in LIGNUM.

This study presents a formal way to model architectural development of
trees in LIGNUM using L-systems  with the language L.  Essentially the
use of L is based on the similarity how bracketed L-systems and LIGNUM
represent the  branching structure of  trees (Section \ref{sec:pine}).
Similar conversions  between modeling  frameworks and tools  have been
reported for example by [Ferraro et al 02] and [Dzierzon et al 03]. In
fact already  [Kurth 94] has reported convergence  between tree models
produced in  AMAP, CIRAD and  the same models expressed  in L-systems.
Also, a large body of  mathematical forms known as fractals describing
plant structures  have been described  with L-systems [Kurth  99]. The
Eq. \ref{eq:ca110} is an  example of one dimensional cellular automata
(CA)  system, the rule  110 in  Wolfram's [Wolfram  02] classification
(the  first eight rules  implement the  CA system  and the  ninth rule
expands  the string).   Interestingly, according  to [Wolfram  02] the
rule 110 is capable of universal computation.

L-systems provide good scientific abstraction needed in plant modeling
(c.f.   [Regev and Shapiro  03]).  An  L-system captures  the relevant
properties of  the phenomena in  its set of symbols  highlighting only
essential characteristics  of the model.  It  is computable supporting
qualitative and quantitative reasoning of the model properties.  It is
extensible, new  symbols can capture additional features  of the model
if required  and it is  understandable, the formal notation  allows to
share, compare  and correct scientific knowledge. For  example part of
an  L-system model  in itself  can appear  in a  publication  as model
description  unlike   models  implemented  with   general  programming
languages.

The  theoretical  advancements later  implemented  in  tools based  on
L-systems formalism  has been motivated  by the curiosity to  find out
what  phenomena  in  plant  modeling  can be  formally  described  and
simulated.   The range  of circumstances  where L-system  formalism is
applicable  is quite  extensive [Prusinkiewicz  99].  In  fact,  it is
possible  to  implement  many  physiological  models  in  LIGNUM  with
L-systems. Section  \ref{sec:vi} gives an  example of the  vigor index
with the language L and it is trivial to implement also current models
for  photosynthesis and  respiration.  With  a little  effort  one can
surely  outline  a  rule  set  for the  iterative  allocation  of  net
photosynthates  to  growth  [Perttunen  et  al 96,  98  and  01].  The
simulation of  plant communities has  been reported by [Kurth  00] and
with  multiset L-systems  developed  by [Lane  and Prusinkiewicz  02].
Inevitably  one  has  to  ask  why not  reimplement  LIGNUM  and  make
L-systems the basis of its future development?

Plants are not closed systems  but interaction with environment has an
important function  in their  development.  Such modeling  include for
example computation of light regime in plant community and competition
for growth  space in clonal  plant. Section \ref{sec:bearberry}  is an
example  of the latter  using algorithm  for collision  detection.  To
express such  phenomena, [M\^ech 97]  and [Mech and  Prusinkiewicz 96]
have extended  L-system formalism with communication  symbols that can
pass  parameter values  between  the  plant model  in  L-system and  a
separate  program simulating for  example relevant  characteristics of
its  environment.  [Kurth  94]  has implemented  a  set of  predefined
functions  to  return environmental  information  to  the L-system  in
GROGRA simulation  program.  Once implemented  these separate programs
or predefined functions  are easily used and reused  but note the time
spent  to implement  these  environmental models,  also called  global
sensitivity [Kurth ??],  is the time spent outside  the formalism with
some general purpose programming language.

Section \ref{sec:vi}  for vigor index  presents how elegantly  one can
express  computations in L-systems.   But there  are two  things worth
noting.  First,  recall the parallel semantics in  rewriting.  Given a
single processor machine  it takes $n$ scans of  the string length $n$
to pass  the symbol $V$ from  the base of the  tree to the  tip of the
branches. Again,  use of global  functions may reduce  the algorithmic
complexity from  $O(n^2)$ to linear  time $O(n)$.  In general  one can
avoid these computational pitfalls  by assuming left to right scanning
of  the string  and arranging  computations using  for  example global
variables. In this  case this might be difficult  because each segment
may have its own unique value for vigor index.  Second, each branching
pattern requires own  rule to match the context.   The example in Fig.
\ref{fig:vi} is for two segments for  forking off but one must be able
to capture one, two, three or any number of branches required.

An essential feature of the  original L-systems theory is its discrete
character.   Symbols are  created, rewritten  and deleted  in distinct
events  captured  by  the  rules.   This limitation  was  assessed  in
differential  L-systems  [Hammel  96]  where modules  or  symbols  are
created and disappear in discrete time steps but develop in continuous
fashion described  by differential equations. But there  is still much
discreteness that remains.

An  important   class  of  chemical   reactions  can  be   modeled  as
reaction-diffusion equations. Suppose a  sequence of tree segments and
two  nutrients $A$  and $B$  present  in the  segments denoting  their
concentrations with $a$  and $b$.  Assume that the  amount of nutrient
$A$ depends  on the quantities of  $A$ and $B$ already  in the segment
and similarly for $B$.  This is  the reaction part of the system. Then
assume  the  two nutrients  can  dissolve  between adjacent  segments.
Typically if in a segment the concentration of the nutrients is higher
than  its neighbors  then the  segments' concentration  will  even out
because of  flow of  the nutrients away  from the  peak concentration.
This  is the  diffusion part  of the  system.  Formally  one typically
writes  $\frac{\partial a}{\partial  t} =  F(a,b) +  C_a\nabla^2a$ for
$A$'s changes  in time and  $\frac{\partial b}{\partial t} =  G(a,b) +
C_b\nabla^2b$ for  $B$.  The Laplacians $\nabla^2 a$  and $\nabla^2 b$
(divergence of  the gradient of the twice  differentiable function) in
diffusion terms are the measure of the concentrations in one location.
Thus one can imagine a continuous  flow of nutrients in every point in
segments  in a  continuous  time expressed  with partial  differential
equations and perhaps design L-system formalism around it but how does
one write  rules having  no explicit time  steps?  In  general partial
differential  equations  are considered  difficult  to  solve and  the
knowledge  of the problem  domain is  needed for  efficient solutions.
Reaction-diffusion model  in the context of  branching structures (and
LIGNUM) is in [Palovaara 03]. Herranjestas ett� kynnet lipsuu.

It is still an open question for us if the modeling effort with LIGNUM
will switch to L-systems or shall  we use our 'mixed approach. In any
case it would be an  interesting exercise in itself to study carefully
the convergence of LIGNUM and L-systems.

