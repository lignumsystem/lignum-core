\section{Introduction to language L}

The  L-system  formalism  is   the  foundation  on  which  programming
languages  for  plant modeling  tools  have  been  built. Though  they
usually include constructs  from general purpose programming languages
the key  concept of  these tools  is rewriting. The  language L  is no
exception.

The counterpart  of a symbol in  L-system formalism is  the concept of
module in the language L. A module  has a name and can take any number
of arguments  of any  type (in C++).   A syntactically  correct module
definition consists of a predecessor defining one module possible with
its  context ending  with colon  and a  production  defining successor
string embraced  within curly braces.   For example the third  rule in
Eq. \ref{eq:ca110} is written in L as:
\begin{equation}
 O() < X() > O(): \{\mathrm{produce}\ X();\}
\end{equation}

For branching structures of plants L predefines two modules $SB()$ and
$EB()$ denoting begin and end  of branch respectively.  To control the
movements of the  turtle (the drawing engine) let  us first define its
orientation  in space by  three unit  vectors $\vec  H$, $\vec  L$ and
$\vec U$ perpendicular to each other such that $\vec U = \vec H \times
\vec L$ denoting turtle's heading,  direction to the left and up. Then
define module $F(s)$ to move the turtle forward along its heading step
of  length  $s$  and  three modules  $Turn(\alpha)$,  $Pitch(\alpha)$,
$Roll(\alpha)$ to  rotate orientation of  the turtle around  $\vec U$,
$\vec  L$  and $\vec  H$  by  $\alpha$.   One special  module  $Start$
corresponds the axiom.

\section{Scots pine: integrating L and LIGNUM}\label{sec:pine}
The  first example  mimics the  growth  of young  Scots pine.   First,
introduce module  $B(O,L)$ representing a bud.   The module definition
implements development for the  main stem and the branching evolution.
In each iteration the apical bud ($O = 0$) moves forward of length $L$
and forks off four new side  branches.  The first order branches ($O =
1$)  are  similar  but  only  two  more  side  branches  are  created.
Branching stops after third order  ($O > 3$).  The definition for this
pine growth in L is in Fig.  \ref{fig:L1}. The development starts from
one bud and  the string after two iterations  is in Eq. \ref{eq:pine2}
(we postpone the  explanation of open and close  statements). Symbol [
denotes $SB()$ and ] denotes  $EB()$, modules for turtle rotations are
not shown:
\begin{equation}\label{eq:pine2}
F\;[F[B][B]B]\:[F[B][B]B]\:[F[B][B]B]\:[F[B][B]B]\; F \;[B][B][B][B]\; B
\end{equation}

We can now outline straightforward  algorithm to convert the string of
symbols in L to structural units in LIGNUM using Eq. \ref{eq:pine2} as
a particular example.   To start with, interpret the  symbol $F(s)$ as
tree segment  of length $s$.  The  symbol B corresponds  to bud.  Each
consecutive  set of $n$  branches between  two $F$  symbols will  be a
branching  point with  $n$ axes.   The string  in  Eq.  \ref{eq:pine2}
becomes first  the main axis  with two segments, two  branching points
and the terminating bud:
\begin{equation}
[TS, BP, TS, BP, B]
\end{equation}
The two branching points contains four axes each:
\begin{equation}
[TS, [A,A,A,A], TS, [A,A,A,A], B]
\end{equation}
Finally, recursively constructing the axes we get:
\begin{equation}
[TS, [TS,[[B],[B]],B],\ldots, [TS,[[B],[B]],B]], TS, [[B],[B],[B],[B]], B]
\end{equation}
Current status of  the turtle updated according to  turtle commands in
the  string   defines  the  position  and  orientation   of  the  tree
compartments in LIGNUM.

The structural development of the tree described in L does not require
the generation  of LIGNUM representation  of trees from  scratch after
each derivation.   In our case we  can assume that  an L-system indeed
generates alternating sequence of  tree segments, branching points and
terminating buds  and the terminal  buds only generate  new structural
units.   Thus  after each  derivation  it  is  possible to  match  the
existing  string   and  LIGNUM  representation  and   insert  the  new
structural units in  the axis lists before the  terminating buds whose
positions and  orientations will be  updated.  The development  of the
pine  after  one,  four  and   eight  development  steps  is  in  Fig.
\ref{fig:pine}.



