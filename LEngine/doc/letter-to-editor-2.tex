\documentclass{letter}
\address{Finnish Forest Research Institute\\Vantaa Research Centre \\ 
         PL18 \\ FIN-01301 Vantaa \\ Finland \\ tel: +358 0 010 211 2328 \\ 
         fax: +358 0 010 211 2203}
\signature{Jari Perttunen \\ jari.perttunen@metla.fi \\ 
           Corresponding author}


\begin{document}
\begin{letter}{ Ecological Modelling \\ Editorial Office \\
               P.O. Box 181 \\ 1000 AD Amsterdam \\ The Netherlands} 
\opening{Ms. K. Fowler,}

Please find the  enclosed manuscript \textsl{Incorporating Lindenamyer
  systems     for     \mbox{architectural}     development    in     a
  functional-structural tree model}, reference number ECOMOD 1697, for
publication  in Ecological  Modelling as  an original  research paper.
This is the second, revised \mbox{submission} of the manuscript.

We have taken  into account and responded to the  comments made by the
two anonymous reviewers and the Editor. More precisely:
\begin{enumerate}
\item  Reviewer \#1
\begin{itemize}
\item The abstract  has the sentence regarding feedback  from the FSTM
  to the L-system.
\item Sections 3 and 4 are now merged into a single section 3. We have
  shortly defined the main  concepts in L-systems including the notion
  of string in the first paragraph  of the section. The example in the
  section is presented with L notation.
\item We thank the reviewer for pointing out typing errors (an obvious
  reason for  confusion) in equations  now numbered 2,3,5 and  8.  The
  use of brackets and commas are now corrected. 
\item We  have explained the  form of integration  of L and  Lignum in
  section  4.2, the  last paragraph.  L+C  or open  L-systems are  not
  involved in the implementation (see Introduction, paragraph 7).
\item The section for namespace is removed.
\item  L (and  L+C)  allows the  mixing  of L-systems  and C/C++  (see
  section 4.2, paragraph 5 and Discussion).
\item We  thank the reviewer for suggested  language corrections. They
  have been taken into account, most notably: 
\begin{itemize}
\item The paragraph in  the introduction using ambiguosly ``combining
  in fact'' when ``linking'' process-based and architectural models is
  removed and we  now state the importance of  developing plant models
  where  architecture  and  functioning  interact  (see  also  changes
  required by the Editor).
\item First  sentence of the last  paragraph of the  first section was
  unclear  for the  reviewer. We  now use  the phrase  ``based  on the
  likeness   of   how  LIGNUM   and   L-systems  represent   branching
  structures''.
\item Regarding  Scots pine  example it is  now mentioned in  the text
  (Section 4, second paragraph) that the A records the branching order
  and that the extension continues after branching stops.
\item The second sentence in the Figure 2 is now in the text.
\item Appendix A the last rule  for Bend the argument for Pitch is now
  's' as it should.
\end{itemize}
\item We  have not labelled  the axis in  Figure 1. We think  the axis
  should be clear in the text and in the equations.
\item We gratefully thank the  reviewer for pointing out the errors in
  references. We have corrected the mistakes and falsehoods.
\end{itemize}
\item Reviewer \#2
\begin{itemize}
\item We  have removed the  sentence proposing ``the optimal  way'' to
  implement complicated plant models.  We now simply motivate our work
  in Discussion, paragraph 6.
\item  The  collision  is  detected  (Section  4.4,  paragraph  4)  if
  $\cos(\alpha/2)  \leq  \frac{{\vec  D}  \cdot {\vec  {P_3}}}  {|\vec
    D||\vec  {P_3}|}$   (as  the  cosine  increases   when  the  angle
  decreases) and $|\vec {P_3}| < l$.
\item We  thank the reviewer for suggested  language corrections. They
  have been taken into account.
\end{itemize}
\item The Editor
\begin{itemize}
\item The Editor asked us to  draw ``the front line'' of modelling and
  point out the  advantages of our model (compared  with other models)
  to  create complex  plant  models.  We  have improved  Introduction,
  especially paragraphs  4 and 5,  where we also explicitly  state the
  importance  of  plant  models,  where architecture  and  functioning
  interact. See also Discussion,  paragraph 8, where one direction for
  future work is suggested.
\end{itemize}
\end{enumerate}

The  manuscript  is typesetted  with  \LaTeXe\  using  style files  by
Elsevier: \textsl{elsart}  document class file, \textsl{template-harv}
template   file,    and   \textsl{elsart-harv}   BiBTeX    file   with
\textsl{natbib} package.
\newpage
The files for the manuscript are in the CD in the ECOMOD1697 directory:

\begin{enumerate}
\item  lsystems-elsart.tex: the  preamble by  Elsevier  for Ecological
  Modelling  including the  abstract and  organising the  rest  of the
  manuscript.
\item manuscript.tex: the manuscript itself.
\item references.bib: the bibilographic database.
\item figure1.tex: figure 1 caption.
\item figure2.tex: figure 2 caption.
\item figure3.tex: figure 3 caption.
\item appendix-A.tex: appendix A.
\item appendix-B.tex: appendix B.
\item figure1-A.eps: figure 1 left.
\item figure1-B.eps: figure 1 middle and right.
\item figure2-A.eps: figure 2 left.
\item figure2-B.eps: figure 2 middle.
\item figure2-C.eps: figure 2 right.
\item figure3-A.eps: figure 3 left.
\item figure3-B.eps: figure 3 middle.
\item figure3-C.eps: figure 3 right.
\end{enumerate}

On Unix  or Linux to reproduce  the manuscript, mount the  CD and copy
the  ECOMOD1697 directory  to your  working directory.   Then  in your
working directory  go to  ECOMOD1697 directory and  type \textsl{latex
  lsystems-elsart},   then  \textsl{bibtex  lsystems-elsart}   on  the
command  line.   Repeat  the  two  commands until  no  warnings  about
references are generated.

\closing{Yours sincerely,}
\encl {1. Two hardcopies of the manuscript. \\
       2. CD containing the manuscript ready for \LaTeXe\.\\
       3. Letter with comments by two reviewers and the Editor-in-Chief.}
\end{letter}


\end{document}
