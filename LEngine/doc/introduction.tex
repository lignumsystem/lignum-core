\section{Introduction}
The research group at the  Department of Forest Ecology, University of
Helsinki has since early  1970's monitored and modelled functioning of
trees and their environment [Hari 99].  Analysis of the collected data
originally focused  on understanding metabolism of trees  but the work
has extended to analyze growth and development of individual trees and
forest stands.

The precursory modelling work [Hari  et al 82] for Scots pine presents
the framework for the stand  models based on process-based single tree
models where the  growth of trees is computed  from photosynthesis and
respiration.  The light regime in  a forest stand is the driving force
of  photosynthesis  in  a  tree  from  which  dry  matter  production,
maintenance  and allocation to  different tree  compartments (foliage,
stem, branches, root system  etc.)  represented as state variables are
contrived using the principle of carbon balance.

The  further analysis  of this  modelling approach  has  improved tree
models  and  lead to  deeper  understanding  the  mechanics of  carbon
allocation,  structural  constraints controlling  tree  form and  tree
interactions supporting the further development of stand models.

To  mention a few,  [M�kel� and  Hari 86]  describe tree  growth using
photosynthetic  light   ratio  to  measure   competitive  ability  and
differentiation of  trees.  [Nikinmaa  92] has optimization  model for
photosynthesis  for  single  tree  involving light  extinction  within
canopy and construction cost  branches; stand level dynamics considers
nutrient   flows  explicitely.    [Siev�nen  93]   studies  structural
relationships  i.e.  pipe  model  for stem,  branches  and foliage  to
express   tree  and   stand   growth  in   terms  of   height-diameter
relationship.   Tree  and  stand   model  in  [M�kel�  97]  represents
geometric dimensions of woody organs  of stem, branches and roots with
constant form parameters.  [M�kel� et al 97] integrates two additional
submodels to  calculate these  parameters explicitely.  The  result is
representation of a three dimensional structure of a tree stem used in
wood quality (branchiness) analysis of Scots pine.

LIGNUM  is  a single  tree  model [Perttunen  et  al  96] faithful  to
modelling approach described (see e.g.  [Nikinmaa 92, Siev�nen 93 and
M�kel�  97])  but instead  of  aggregated  tree  parts it  has  three
dimensional  description  of  the  above  ground  part  of  the  tree.
Following  the principles  in  [Hari  et al  82].   LIGNUM includes  a
detailed model  of self-shading within  a tree crown [Perttunen  et al
98,  Perttunen  et  al  2001]  from which  the  radiation  regime  for
photosynthesis in different parts of the tree can be computed.  If the
photosynthates produced exceed respiration costs the net production is
allocated to new and existing tree structure acquiring carbon balance.
LIGNUM has been applied for both coniferous [Perttunen et al 96, Lo et
al 2000] and broad leaved trees [Perttunen et al 2001].

The  main  focus  in  LIGNUM   has  been  to  develop  the  functional
(metabolic)  part of  the model.   Until now  there has  not  been any
method  to formally define  the architecural  development of  the tree
structure.   Research has  produced remarkable  results in  this field
however.   A leaf  of Barnsley's  fern  [Barnsley 88]  is a  prominent
example  of  fractal theory  providing  a  way  to describe  the  self
similarity of natural objects.

[Halle  et  al  1970]  have  studied tropical  botany,  analysed  tree
structure  and produced  25  architectural models  based on  branching
pattern, senescense and functioning  of buds.  Based on Halle's models
de  Reffye  [overview  de Reffye  et  al  88,  90, 97]  has  developed
botanically sound automaton theory which has been put into practice in
AMAP  software [Blaise 98].   de Reffye's  automaton theory  have been
further  developed by  [Yan  et  al 01]  where  a plant's  topological
structure is  composed from its subparts  avoiding possibly repetitive
and  computatinally  costly  internode  by internode  construction  of
trees.

Lindenmayer-systems (L-systems)  invented 1968 by  Aristid Lindenmayer
[Lindenmayer 68,71]  were initially meant to  describe the development
of  multicellular organisms.  L-systems  are string  rewriting systems
and their research  is concerned what phenomena can  be described with
formal  languages.   Prusinkewicz,  Kurth  and other  scientists  have
developed the  theory, tools and applications for  modeling plants and
their  environment using L-systems  framework.  The  ubiquitous theory
and the progress in L-systems  in modelling plants and trees until the
end of '90's is well documented in [Prusinkiewicz and Lindenmayer 96],
[Prusinkiewicz 99] and [Kurth 99].

The objective of  this study is to enhance  the modelling capabilities
of  LIGNUM with L-systems  using a  subset of  language L  to formally
describe architectural development of trees.  We clearly recognize the
original design and implementation of the language L to [Prusinkiewicz
and Karwowski??].  Here we present the use of L in LIGNUM based on the
similarity  how  LIGNUM and  bracketed  L-systems represent  branching
structures  of  trees  [Perttunen   et  al  96,98].   We  give  sample
applications  of   the  system  and  discuss  our   approach  for  its
consequences to  the future development  of LIGNUM to model  trees and
forest stands.

