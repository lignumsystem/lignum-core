\section{Introduction}

Functional-structural tree models (FSTM) try to determine dynamics and
growth of woody perennial  plants by assessing physiological processes
(function)  in their three  dimensional arborescent  form (structure).
Physiological  processes  or  plant  metabolism involves  for  example
photosynthesis of  the foliage  using the radiation  regime encircling
the  plant, respiration,  flow  of water,  nutrients  or hormones  and
allocation of  nutritive substances.  The  three dimensional structure
of the  tree and its  changes in time  should be modelled in  terms of
detailed  botanical or  morphological  units rather  than with  highly
aggregated plant compartments.

The  term 'functional-structural  model' was  introduced  in [Silva97]
that  compiles a series  of articles  extending the  traditional three
step analysis  of an entity  (e.g.  a plant)  where the nature  of the
intrinsic  processes  of the  entity  are  analysed, the  differential
equations describing  the processes  are inferred and  solved possibly
assuming some  reasonably sufficient geometry for the  entity. The key
idea in the  various models presented in [Silva97]  was that the focus
was set  on what  happens in  a single plant  element (such  as shoot,
internode, leaf,  root hairs  etc.)  and defining  their interactions.
Obviously,  due to  thousands  of interacting  units  such models  are
implemented as computer  programs where the program acts  as the model
and as the means of simulating the model behaviour.

Winfried  Kurth  has   classified  existing  modelling  approaches  in
forestry  into   three  categories  and  based   on  this  observation
constructed a model triangle.  The  top apex of this triangle is taken
up by aggregated  stand models.  Process based single  tree models and
architectural tree  models are placed at  the two other  apices at the
base of the  model triangle respectively.  Moving along  the sides and
inside  the triangle  captures the  model continuum  including aspects
from one or the other two modelling avenues.

Consequently there are practically two  ways to construct an FSTM. One
can start from an architectural model adding functional, physiological
details into it.  The second approach is to begin with a process based
physiological model and extend it with structural details.

Examples
-AMAP 'French School'
-'Helsinki School'
-'L-System School'

By  looking  The  two  perspectives  of  plant  modelling,  structural
dynamics  and  physiological  processes,  have  evolved  independently
simplifying  or  simply  neglecting   the  standpoint  of  the  other.
However, the progress  in these two lines of  modelling have gradually
involved characteristics of each other bringing them closer.



LIGNUM is an FSM applied to 
Current interest

Objectives of this study.
 

