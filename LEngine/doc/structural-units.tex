\section{Structural units  of LIGNUM} 

LIGNUM  is intended   as a  generic   model  for  both coniferous  and
broad-leaved trees.  Different   tree   species can be   simulated  by
implementing  models  of  metabolism,  structural  dynamics of  birth,
growth and senescence, and by  realizing branching rules for  distinct
tree   architectures   \citep{perttunen:96,  perttunen:01}.   Here  we
briefly present the main features of the model  in order to understand
how LIGNUM is adapted to use L systems.

LIGNUM represents  the three-dimensional aboveground part  of the tree
with  four  structural  units  called  tree  segment  (TS),  bud  (B),
branching point  (BP) and  axis (A).   A branching point  is a  set of
axes. An  axis is  a sequence of  tree segments, branching  points and
terminating bud.   This captures the  recursive structure of  the tree
crown.

The most  important functioning  unit is the  tree segment,  i.e.  the
section of woody material  between two branching points.  For conifers
the needles  are currently modeled  as a cylindrical layer  of foliage
surrounding  a  tree   segment  (Fig.   \ref{fig:model},  left).   For
broad-leaved  trees  leaves  are  attached  to  recently  formed  tree
segments  and are modelled  explicitly using  a simple  geometric form
such as an ellipse to  represent the leaf shape (Fig. \ref{fig:model},
middle).

An  axis is implemented  as a  list.  Adopting  the notation  from our
previous work \citet{perttunen:96}, let  the left bracket ('[') denote
the beginning of  the list, the right bracket (']')  denote the end of
the list, and the list  elements are separated by commas. For example,
the main  axis of the  model tree for  a coniferous species  in Figure
\ref{fig:model},  consisting of three  tree segments,  three branching
points and the terminating bud, writes:

\begin{equation}
[TS_0,BP_1,TS_2,BP_3,TS_4,BP_5,B_6]
\end{equation}

The  indices denote  the positions  of the  elements in  the  list.  A
branching point is implemented as a list of axes.  The three branching
points in the model tree  contain two axes (i.e.  branches) each. Thus
the main axis can be written:

\begin{equation}\label{eq:mainax}
[TS_0,[A,A],TS_2,[A,A],TS_4,[A,A],B_6]
\end{equation}

Each of the  four axes in the first two branching  points consist of a
tree  segment, branching point  and bud.   The two  axes in  the third
(last) branching point  contain one bud each. When  these are inserted
into Eq.  \ref{eq:mainax},  the structure of the whole  model tree can
be expressed as (omitting the position numbering):

\begin{equation}\label{eq:tree}
[TS,[[TS,[],B],[TS,[],B]],TS,[[TS,[],B],[TS,[],B]],TS,[[B],[B]],B]
\end{equation}

The  empty lists ([]) for  branching points denote  no forking off
axes, thus maintaining the structural integrity of the model.

