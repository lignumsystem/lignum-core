\section{Structural units of LIGNUM}
LIGNUM is intended as a generic modelling tool for both coniferous and
hardwood trees.  Different tree  species can be simulated by assessing
descriptions of  metabolism, structural dynamics of  birth, growth and
senescence,  and by  implementing  branching rules  for distinct  tree
architectures  \citep{perttunen:96,  perttunen:01}.   Here we  present
shortly the main model features to understand how LIGNUM is adapted to
use L-systems.

The metabolic  activities incorporated in LIGNUM  are interception and
attenuation  of  photosynthetically  active  solar radiation  in  tree
crown,  photosynthetic  production,  respiration and  partitioning  of
growth  between  new parts  and  existing  structure. Some  mechanical
interactions,  like bending of  branches or  collision of  branch tips
have also been implemented.

LIGNUM represents the three-dimensional  above ground part of the tree
with  four  structural  units  called  tree  segment  (TS),  bud  (B),
branching point (BP) and axis (A). The most important functioning unit
is the cylindrical tree segment, section of woody material between two
branching points.  For  conifers needles are at the  moment modeled as
cylindrical  layers   of  foliage  surrounding   tree  segments.   For
deciduous species leaves attached  to tree segments are considered and
studied explicitly  using simple geometric  form like ellipse  as leaf
shape.  Branching  point and axis  capture the recursive  structure of
the tree  crown.  Branching point is  a set of  axes and an axis  is a
sequence of tree segments, branching points and terminating bud.

An axis  is implemented as  a list. For  example the main axis  of the
model tree for coniferous species (\ref{fig:model} ) consists of three
tree segments, three branching points and the terminating bud:

\begin{equation}
[TS_0,BP_1,TS_2,BP_3,TS_4,BP_5,B_6]
\end{equation}

A branching  point is implemented as  list of axes. In  the model tree
the branching points contain two axes  each. Thus the main axis can be
written:

\begin{equation}
[TS_0,[[A,A]],TS_2,[A,A],TS_4,[A,A],B_6]
\end{equation}

Finally, the structure of the whole model tree is expressed as:

\begin{equation}\label{eq:tree}
[TS_0,[[TS,[],B],[TS,[],B]],TS_2,[[TS,[],B],[TS,[],B]],TS_4,[[B],[B]],B_6]
\end{equation}

The two empty  lists ([]) for branching points  denote no axes forking
off maintaining the structural integrity of the model.

