\documentclass[12pt,a4paper,english]{article}
%Scandinavian alphabet
\usepackage{babel}
\usepackage{t1enc,epsfig,textcomp,supertabular}

%Citation style 
\usepackage[round,authoryear]{natbib}
\hyphenation{Linden-mayer meta-bolism} 

\title{Enhancing the Modeling Capabilities of LIGNUM with Lindenmayer systems}

\author{Jari Perttunen
        \footnote{Corresponding author, email: jari.perttunen@metla.fi.}
        \footnote{Finnish Forest  Research Institute, Vantaa Research
                  Station, P.O.Box 18, 01301 Vantaa, Finland.}
\and Radek Karwowski\footnote{University of Calgary}
\addtocounter{footnote}{-1}
\and Przemyslaw Prusinkiewicz\footnotemark
\addtocounter{footnote}{-3}
\and Risto Siev�nen\footnotemark
}


%suggested running title
\markright{Perttunen et al. -- Lindenmayer systems in LIGNUM}
\pagestyle{myheadings}
%Double line space if needed
%\linespread{1.65} 

%This is how to get around nonnumerical years (xxxxa,xxxxb, etc.)
%with natbib when the author has been really energetical. 
\defcitealias{kurth:em94}{Kurth,~1994b}

%Equation numbers in brackets
%\makeatletter
%\renewcommand\@eqnnum{\hb@xt@.01\p@{}%
%                      \rlap{\normalfont\normalcolor
%                        \hskip -\displaywidth[\theequation]}}
%\makeatother

\begin{document}

\maketitle
\pagebreak
\abstract{

LIGNUM is  a functional-structural tree  model that represents  a tree
with  four modelling units  closely resembling  the real  structure of
trees.  These are  tree segments, the tree axes,  branching points and
buds.  Metabolic  processes are  explicitly related to  the structural
units  in  which they  are  taking  place.   Here we  enhance  earlier
versions  of  LIGNUM  for  conifers  and broadleaved  trees  with  the
possibility to  formally describe  the architectural development  of a
tree with Lindenmayer systems.

This  modified version  of  LIGNUM  is then  used  to demonstrate  its
potential capabilities  to model  not only single  trees but  also the
development   of   forest    stands.    The   current   implementation
methodologically mixes the term  rewriting of Lindenmayer systems with
the LIGNUM model  to represent a tree as an  abstract datatype and the
functioning of  a tree with  generic programming. Using  some examples
(from  the  previous and  future  work  with  LIGNUM) we  discuss  our
approach  identifying from our  point of  view some  important missing
pieces  in L-systems  to  present  them as  an  overture for  L-system
community for possible theoretical and practical assessment.

\bf Key words\rm: LIGNUM, Lindenmayer systems, functional-structural tree model.  
}

\section{Introduction}
\section{The Language L in LIGNUM}
\section{Examples}
\section{Discussion}
%References 
%\bibliography{references/text-books,references/misc,references/em98,references/sugar-maple,references/aob96}
%Annals of Botany bibliographic style
%\bibliographystyle{plain}
%Tables and figures.
\end{document}


