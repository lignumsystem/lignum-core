\chapter{Getting Started with LIGNUM}

\section{Compiling LIGNUM}

The  program implementing  \lignum\  is written  with  C++ under  UNIX
environment. This guide tells how to compile and run the core model of
\lignum\  under Silicon  Graphics Irix  6.x machines with native SGI
$ \rm MIPSpro^{TM}$ C++ 7.x compilers.  As \lignum\ is ported to other
machine  architectures  they  will  be  documented  as  well.   


The source  code implementing  the core model  of \lignum\  called LGM
(libLGM.a) is located in stl-lignum directory. The LGM library depends
on c++adt  library (libc++adt.a) implementing  general purpose classes
not found in Standard Template Library.  The source code for c++adt is
in c++adt directory. The c++adt  and stl-lignum directories must be in
the same directory level:

\begin{tabbing}
Tabbing\= Tabbing\=\kill
\>\>prompt\% ls -F \\
\>\>c++adt/ stl-lignum/
\end{tabbing}

Before  the compilation also  the Standard  Template Library  (STL) is
needed. The STL library should be part of the C++ compiler system.  To
compile both the c++adt and  the LGM libraries simply go to stl-lignum
directory and type make:

\begin{tabbing}
Tabbing\= Tabbing\=\kill
\>\>prompt\% cd stl-lignum \\
\>\>prompt\% make
\end{tabbing}

The main  Makefile for the make  utility to build the  LGM library not
only compiles the LGM library but also assumes the existence of c++adt
directory and if needed (re)compiles the c++adt library.

\section{Makefiles for LIGNUM}

The make utility  is used as an integral  part of software development
in Unix environment.   A description file for the  make utility called
Makefile contains  a set of macros  and a set  of specifications. Each
specification  consists of  a target,  optional prerequisites  for the
target  and optional commands  to be  executed by  the utility  when a
prerequisite is newer than the target.

\subsection{Predefined Macros}

There are several Makefiles that are used to compile the core model of
\lignum\ and the c++adt library.   To  keep  these  two
libraries consistent,  e.g., to compile for  the 32 or  64 bit machine
architectures some useful common macros are predefined.

\begin{enumerate}
\item CCC: The compiler. The default value is CC -64 to compile 64 bit code.
\item CCCFLAGS: General flags for the compiler. The default value is 
 -c -ptv -prelink to be verbose (-ptv) when compiling templates.
\item OPTIMIZE: Flags to optimize the object code. The default value
is -g.
\end{enumerate}
For example to maximize the performance for 64 bit code for IP28 target platform type:
\begin{tabbing}
Tabbing\= Tabbing\=\kill 
\>\>prompt\% make ``OPTIMIZE=-Ofast=ip28''
\end{tabbing}

In general  the the  appropriate selection for  a SGI platform  can be
determined by running  ``hinv -c processor". 




\section{Running LIGNUM}

\subsection{Initialization of the Tree}

The parameters and functions of the tree for the simulation 
must be given in files. The file that is given as command 
line argument is the schema file i.e., the file that gives the 
names of files where the parameters and functions are defined, 
not the parameters and functions themselves. An example of the schema file
is below:

\begin{verbatim}
#File describing the location of definition files
#for parameters and fuctions of LIGNUM
Parameters:
    Tree: Tree.txt 
    Firmament: Firmament.txt
Functions:
    FoliageMortality:FoliageMortality.fn
    Buds: Buds.fn
    DoI: DoI.fn
\end{verbatim}

The format of the schema file is simple. For the first, the \tt \# \rm
character starts a comment that extends to the end of line.  

The schema file has two main sections, one for parameters and
the second one for functions denoted by keywords 
\tt Parameters \rm and \tt Functions \rm followed by a colon.
 
The sections for parameters and functions consists pairs
composed by a keyword,(e.g., \tt Tree \rm and 
\tt Foliage Mortality\rm) and a file name
(\tt Tree.txt \rm and \tt FoliageMortality.fn \rm respectively). 
The keyword tells the purpose of the file, so that
during the intialization parameters and functions are
intialized properly. The colon is used to separate the
keyword from the file name.

If necessary, the the sections for functions and parameters
will be subdivided in the future. 

An example of the parameter file for the tree is below.

\begin{verbatim}
#Parameters for tree compartments according to papers in
#Annals of Botany 1996 and in Ecological Modelling 1998.

af      1.30        #Needle mass-tree segment area (kg/m^2)
                    #relationship
ar      0.50        #Foliage - root relationship 
lambda  1.3         #Intial value for lambda
lr      100.0       #Length - radius relationship of a 
                    #tree segment
mf      0.20        #Maintenance respiration rate of foliage
mr      0.240       #Maintenance respiration rate of roots
ms      0.0240      #Maintenance respiration rate of sapwood
na      0.7854      #Needle angle (pi/4)
nl      0.10        #Needle length (10 cm = 0.10 m) 
q       0.10        #Tree segment shortening factor
sr      0.330       #Senescence rate of roots
ss      0.07        #Senescence rate of sapwood
rho     400.0       #Density of wood in tree segment
pr      0.0010      #Proportion of bound solar radiation
                    #that is used in photosynthesis
xi      0.60        #Fraction of heartwood in tree segments
\end{verbatim}

The file simply contains parameter value pairs. Each parameter
name is reserverd keyword so that the program can recognize it.
The \tt \# \rm character begins a comment extending to the end of line.

The functions defining different behavior in the tree are given
as parametric curves in ASCII files. See the class 
\tt ParametricCurve \rm in \tt libc++adt.a \rm for details.

The keywords \tt FoliageMortality\rm, \tt Buds\rm, and
\tt DoI \rm denote functions for foliage mortality, number of new buds 
and relative shadiness (degree of interaction) respectively. More functions
will be implemented if necessary.

Currently directory paths are not parsed so all the files,
the schema file and the files defining the parameters and functions,
must be in the same directory where \lignum\ is started.


\section{Visualizing Results}

\subsection{The Visualization Program}
\subsection{MineSet Program}








