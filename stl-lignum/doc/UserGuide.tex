\chapter{Getting Started with LIGNUM}

\section{Getting Lignum}

The development of Lignum is managed by using CVS version control. CVS
has become a popular  system that allows several people simultaneously
to develop a program. To get Lignum type:
\begin{tabbing}
Tabbing\=\kill
\>prompt\% cvs checkout c++adt \\
\>prompt\% cvs checkout stl-lignum
\end{tabbing}
If you must obtain the software over the network type on your machine:
\begin{tabbing}
Tabbing\=\kill
\>prompt\% cvs -d <user>@magnolia.metla.fi checkout c++adt \\
\>prompt\% cvs -d <user>@magnolia.metla.fi checkout stl-lignum
\end{tabbing}
Naturally you must have a user  account on the machine that is the CVS
repository.   In   the  example  the  repository  is   assumed  to  be
magnolia.metla.fi. For efficient use of  CVS see the manual written by
Cederqvist \cite{Ced}.

\section{Compiling LIGNUM}

Before the compilation the c++adt and the stl-lignum must be at the
same directory level:
\begin{tabbing}
Tabbing\=\kill
\>prompt\% ls -F \\
\>c++adt/ stl-lignum/
\end{tabbing}
The STL  library must  also be  part of the  C++ compiler  system.  To
compile both the c++adt and  the LGM libraries simply go to stl-lignum
directory and type make:
\begin{tabbing}
Tabbing\=\kill
\>prompt\% cd stl-lignum \\
\>prompt\% make
\end{tabbing}
To compile the test set for various generic algorithms used in Lignum 
type
\begin{tabbing}
Tabbing\=\kill
\>prompt\% make algorithms
\end{tabbing} 
The main  Makefile for the make  utility to build the  LGM library not
only compiles the LGM library but also assumes the existence of c++adt
directory and if needed recompiles the c++adt library.

\section{Makefiles for LIGNUM}

The make utility  is used as an integral  part of software development
in Unix environment.   A description file for the  make utility called
Makefile contains  a set of macros  and a set  of specifications. Each
specification  consists of  a target,  optional prerequisites  for the
target  and optional commands  to be  executed by  the utility  when a
prerequisite is newer than the target.

\subsection{Predefined Macros}

There are several  Makefiles that are used to compile  the LGM and the
c++adt libraries.  To keep  the libraries consistent (e.g., to compile
for the 32 or 64 bit  machine architectures) and to ease the change of
a  compiler some  useful common  macros throughout  the  Makefiles are
predefined.

\begin{itemize}
\item CCC The compiler.
\item CCCFLAGS The  -c option  must  be present  to tell  the compiler  to
  generate  only object files  (i.e., no  linkage) from  source files.
\item OPTIMIZE Flags to optimize the object code.
\item ARCOMMAND The program to create a program library.
\item ARCOMMANDFLAGS Command line flags for the archive program. 
\end{itemize}
For example on SGI to maximize the performance for 64 bit code for IP28
target platform using inter procedural analysis with default settings type:
\begin{tabbing}
Tabbing\=\kill 
\>prompt\% make ``OPTIMIZE=-Ofast=ip28 -IPA'' algorithms
\end{tabbing}

In general  the the  appropriate selection for  a SGI platform  can be
determined by running  ``hinv -c processor". To study inter procedural
analysis see manual page for ipa. To compile on Linux type:
\begin{tabbing}
Tabbing\=\kill 
\>prompt\% make ``CCC=g++'' ``CCCFLAGS=-c'' ``OPTIMIZE=-O'' \\
\>``ARCOMMAND=ar''  ``ARCOMMANDFLAGS=rv'' algorithms 
\end{tabbing}

There are some other useful Makefile targets defined.
\begin{itemize}
\item html Generate www documentations of the program files. The
enscript program must be installed.
\item changelog Generate ChangeLog file containing the summary of the
cvs logs.
\item clean Remove compiled files.
\item distclean Remove compiled files also in c++adt directory.
\item depend Create file dependencies for LGM library.
\end{itemize}
You may find the Makefile for LGM library useful when starting your
own project. 

\section{Running LIGNUM}

\subsection{Initialization of the Tree}

The parameters and functions of the tree for the simulation 
must be given in files. The file that is given as command 
line argument is the schema file i.e., the file that gives the 
names of files where the parameters and functions are defined, 
not the parameters and functions themselves. An example of the schema file
is below:

\begin{verbatim}
#File describing the location of definition files
#for parameters and fuctions of LIGNUM
Parameters:
    Tree: Tree.txt 
    Firmament: Firmament.txt
Functions:
    FoliageMortality:FoliageMortality.fn
    Buds: Buds.fn
    DoI: DoI.fn
\end{verbatim}

The format of the schema file is simple. For the first, the \tt \# \rm
character starts a comment that extends to the end of line.  

The schema file has two main sections, one for parameters and
the second one for functions denoted by keywords 
\tt Parameters \rm and \tt Functions \rm followed by a colon.
 
The sections for parameters and functions consists pairs
composed by a keyword,(e.g., \tt Tree \rm and 
\tt Foliage Mortality\rm) and a file name
(\tt Tree.txt \rm and \tt FoliageMortality.fn \rm respectively). 
The keyword tells the purpose of the file, so that
during the intialization parameters and functions are
intialized properly. The colon is used to separate the
keyword from the file name.

If necessary, the the sections for functions and parameters
will be subdivided in the future. 

An example of the parameter file for the tree is below.

\begin{verbatim}
#Parameters for tree compartments according to papers in
#Annals of Botany 1996 and in Ecological Modelling 1998.

af      1.30        #Needle mass-tree segment area (kg/m^2)
                    #relationship
ar      0.50        #Foliage - root relationship 
lambda  1.3         #Intial value for lambda
lr      100.0       #Length - radius relationship of a 
                    #tree segment
mf      0.20        #Maintenance respiration rate of foliage
mr      0.240       #Maintenance respiration rate of roots
ms      0.0240      #Maintenance respiration rate of sapwood
na      0.7854      #Needle angle (pi/4)
nl      0.10        #Needle length (10 cm = 0.10 m) 
q       0.10        #Tree segment shortening factor
sr      0.330       #Senescence rate of roots
ss      0.07        #Senescence rate of sapwood
rho     400.0       #Density of wood in tree segment
pr      0.0010      #Proportion of bound solar radiation
                    #that is used in photosynthesis
xi      0.60        #Fraction of heartwood in tree segments
\end{verbatim}

The file simply contains parameter value pairs. Each parameter
name is reserverd keyword so that the program can recognize it.
The \tt \# \rm character begins a comment extending to the end of line.

The functions defining different behavior in the tree are given
as parametric curves in ASCII files. See the class 
\tt ParametricCurve \rm in \tt libc++adt.a \rm for details.

The keywords \tt FoliageMortality\rm, \tt Buds\rm, and
\tt DoI \rm denote functions for foliage mortality, number of new buds 
and relative shadiness (degree of interaction) respectively. More functions
will be implemented if necessary.

Currently directory paths are not parsed so all the files,
the schema file and the files defining the parameters and functions,
must be in the same directory where \lignum\ is started.


\section{Visualizing Results}

\subsection{The Visualization Program}
\subsection{MineSet Program}








