\section{The Class Tree}

The class \tt Tree \rm captures the structure and 
functioning of a single tree. It also allows the user of the class
to query the status of the tree and individual tree
compartments (e.g., after a simulation to to collect data 
for further analysis).

The \tt Axis \rm is the list of tree compartments. 
The adjustable parameters defining the functionig of the different  
tree compartments are collected into  class \tt TreeParameters \rm 
and the attributes describing the status of the whole  tree 
during simulations are in the class \tt TreeAttributes\rm. 

The third group of variables, called transit
variables, are in the class called \tt TreeTransitVariables\rm. 
Transit variables are ``technical''. They don't have a
biological meaning but are needed to proceed one interval of time
(one time step) in a simulation. A transit varible may be needed
for example to find a root of a function.

The class \tt TreeFunctions \rm collects user defined functions needed
in modelling growth and senescence processes of the tree. Each
function is defined as a parametric curve in an ASCII file. 
For the details, see the class \tt ParametricCurve \rm in the library 
\tt libc++adt.a\rm.

\subsection{The Class Declaration of Tree}
\begin{verbatim}
class Tree: public TreeCompartment{
friend void InitializeTree(Tree& tree, 
                           const CString& schema_file);
friend TP GetAttributeValue(const Tree& tree, const TAD name);
friend YEAR GetAttributeValue(const Tree& tree, const TAI name);
template <class T1,class T2>
friend T2 SetAttributeValue(Tree& tree, const T1 name, 
                            const T2 value);
friend TP GetParameterValue(const Tree& tree, const TPD name);
friend TP SetParameterValue(Tree& tree, const TPD  name, 
                            const TP value);
friend TP GetTransitVariableValue(const Tree& tree, 
                                  const TTD name);
friend TP SetTransitVariableValue(Tree& tree, const TTD name, 
                                  const TP value);
public:
  Tree();
  Tree(const Point<METER>& p, const PositionVector& d);
private:
  TreeFunctions tf;
  TreeParameters tp;
  TreeTransitVariables ttp;
  Axis axis;
  RootSystem rs;
};
\end{verbatim}

\subsection{Functions and Operations on the Class Tree}

The functions and operations on the tree are used to intiliaze 
the tree in the beginning of a simulation, querying and setting
values for parameters, attributes and transit variables.

To query or set a value for a parameter (attribute or transit
variable) with the functions available, the name of the parameter
(attribute or transit variable) must be known.  To maintain the
consistency in the program, the name given as an argument to the
function is a symbolic constant of the same name as the parameter in
question. 

For example, the following scenario can be used to check to query if
the parameter value for tree segment - needle area relationship (field
\tt af \rm in the class \tt TreeParameters \rm) is correctly installed.

\begin{tabbing}
Tabbing \= Tabbing \= Tabbing \= \kill
\>\>\> \tt Tree tree; \\
\>\>\> \tt TP af1; \\
\>\>\> \tt InitializeTree(tree,''SchemaFile''); \\
\>\>\> \tt af1 = GetParameterValue(tree,af);
\end{tabbing}

The call \tt GetParameterValue(t,af) \rm returns the value of the
parameter.  

The functions are overloaded and the types parameterized
if necessary to maintain the transparency with different datatypes.

\subsubsection{friend void InitializeTree(Tree\& tree,
               const CString\& \\ schema\_file);}
Initialize the tree by reading parameter values and functions
from in files found in \tt  schema\_file\rm.
\begin{description}
  \item[Arguments] for the function.
    \begin{description}
      \item[tree] The tree.
      \item[schema\_file] The schema file naming
           the files (and their locations) for parameters and functions
           needed for simulation.
    \end{description}
 \item[Returns] Nothing.
\end{description}

\subsubsection{GetAttributeValue();}
Overloaded function to query the value of an attribute 
for the tree (one of the variable fields in the class 
\tt TreeAttributes\rm).

\paragraph{friend TP GetAttributeValue(const Tree\& tree,
                                       const TAD name);}
\begin{description}
  \item[Arguments] for the function.
    \begin{description}
      \item[tree] The tree.
      \item[name] The name of the attribute.
        \begin{description}
          \item[Possible values] One of \tt M, P \rm or \tt Wr\rm.
        \end{description}
     \end{description}
   \item[Returns] The value of the parameter. The type is
                  is a symbolic type definition for built 
                  in double data type.
\end{description}

\paragraph{friend YEAR GetAttributeValue(const Tree\& tree,
                                         const TAI name);}
\begin{description}
  \item[Arguments] for the function.
    \begin{description}
      \item[tree] The tree.
      \item[name] The name of the attribute.
        \begin{description}
          \item[Possible values] \tt age\rm.
        \end{description}
     \end{description}
   \item[Returns] The value of the parameter. The type is
                  is a symbolic type definition for built 
                  in int data type.
\end{description}

To query the age of the tree there one simply types
\tt GetAttributeValue(t,age)\rm. The returned value is the 
age of the tree and the type is a symbolic type definition for 
built in int data type.

\section{The Class Declaration of TreeAttributes}

The class \tt TreeAtrtributes \rm collects attributes (state
variables) of the tree. Eventhough the class is used as a private data
member in the class \tt Tree \rm, access to attributes in a controlled
manner is provided by using overloaded function \tt
GetAttributeValue() \rm declared in the class \tt Tree\rm.

\begin{verbatim}
class TreeAttributes{
public:
  TreeAttributes();
  YEAR age;              
  METER lb;              
  KGC P;                 
  KGC M;                
  KGC Wr;              
};
\end{verbatim}

\section{Public Data Members of TreeAttributes}

\begin{description}
  \item [age] Age of the tree in years.
    \begin{description}
      \item[Unit] --.
    \end{description}
  \item [lb]  Longest branch as a vertical projection 
              from the main stem.
    \begin{description}
      \item[Unit] $m$.
    \end{description} 
  \item [P]   Annual photosynthesis of the whole tree.
    \begin{description}
      \item[Unit] $kgC$.
    \end{description} 
  \item [M] Annual respiration of tree.
    \begin{description}
      \item[Unit] $kgC$.
    \end{description}
  \item [Wr]  Root mass of the tree.
    \begin{description}
      \item[Unit] $kgC$.
    \end{description}
\end{description}

\section{The Class Declaration of TreeParameters}

The class \tt TreeParameters \rm collects adjustable parameters for
all tree compartments. It is a private members of the class \tt Tree
\it but access to the parameters in a controlled manner is providided
by using overloaded function \tt GetParameterValue() \rm declared in
the class \tt Tree\rm.

\begin{verbatim}
class TreeParameters{
public:
  TreeParameters();
  TP af;            
  TP ar;            
  TP lr;            
  TP mf;            
  TP mr;            
  TP ms;            
  TP pr;            
  TP q;             
  TP sr;           
  TP ss;            
  TP rho;          
  TP xi;            

};
\end{verbatim}

\section{Public Data Members for TreeParameters}

\begin{description}
  \item[af] Needle mass - tree segment area relationship.
    \begin{description}
       \item[Unit]  $kgm^{-2}$.
       \item[Default value] 1.3.
    \end{description}
  \item[ar] Foliage - root relationship.
    \begin{description}
       \item[Unit] $kgkg^{-1}$. 
       \item[Default value] 0.5.
    \end{description}
  \item[lr] Length -Radius ratio for a new tree segment.
    \begin{description}
       \item[Unit] --.
       \item[Default value] 1.3.
    \end{description}
  \item[mf] Maintenance respiration rate of foliage.
     \begin{description}
        \item[Unit] $kgCkgC^{-1}year^{-1}$.
        \item[Default value] 0.2.
     \end{description} 
  \item[mr] Maintenance respiration rate of roots.
      \begin{description}
         \item[Unit] $kgCkgC^{-1}year^{-1}$.
         \item[Default value] 0.24.
      \end{description}
  \item[ms] Maintenance respiration rate of sapwood. 
      \begin{description}
        \item[Unit] $kgCkgC^{-1}year^{-1}$.
        \item[Default value] 0.024.
      \end{description}
  \item[pr] Propotion of \it absorbed \rm (bound) solar radiation 
            used in photosynthesis 
       \begin{description}
         \item[Unit] --.
         \item[Default value] 0.001.
       \end{description}
  \item[q] Tree segment shortening factor.
       \begin{description}
         \item[Unit] --.
         \item[Default value] 0.1.
       \end{description}
  \item[sr] Senescence rate of roots.
       \begin{description}
         \item[Unit] $1 \times year^{-1}$.
         \item[Default value]  0.33.
       \end{description}
  \item[ss] Senescence rate of sapwood.  
       \begin{description}
         \item[Unit] $1 \times year^{-1}$.
         \item[Default value] 0.07.
       \end{description}
  \item[rho] Density of wood. 
       \begin{description}
         \item[Unit] $kgm^{-3}$. 
         \item[Default value] 400.
       \end{description}
  \item[xi] Fraction of heartwood in new tree segments. 
        \begin{description}
          \item[Unit] --.
          \item[Default value] 0.6.
        \end{description}
\end{description}

\section{The Class TreeTransitVariables}

By transit variables of the tree it is variables that don't directly
affect the behaviour and growth of the tree but are necessary to make
computations to proceed one interval of time (one time step in
simulation). More formally, let us denote the status of the tree at
one moment of time $ t $ with $ x(t)$, the functions and parameters
for external conditions (e.g., weather) etc. needed to describe the
behaviour of the tree with $ u(t) $ and the transit variables with $
\theta_{i} $ then we can write $ f(x(t),\theta_{1}\ldots\theta_{n},
u(t)) \rightarrow g(x(t)) \rightarrow x(t+1) $ where $ f $ is the
model (implemented e.g. as a computer program).

The class \tt TreeTransitVariables \rm is declared as a private data
member in the class \tt Tree \it but the controlled access is provided
using the function \tt GetTransitVariableValue() \rm declared in the
class \tt Tree \rm.  

\section{The Class Declaration of TreeTransitVariables}
\begin{verbatim}
class TreeTransitVariables{
public:
  TreeTransitVariables();
  TP lambda;        
};
\end{verbatim}

\section{Public Data Members of TreeTransitVariables}
  \begin{description}
    \item[lambda] Variable to balance carbon balance equation.
      \begin{description}
        \item[Unit] --.
        \item[Default value] 1.3.
      \end{description}
 \end{description}









