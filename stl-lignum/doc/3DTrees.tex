\section{Three Dimensional Tree Models}

\section{The Role of Three Dimensional Tree Models}

The study of three dimensional tree models, also
known as virtual trees, is an active
field of research motivated by several reasons. 

For the first, there is always the basic research interested in
finding formal methods describing the architecture of
trees and plants in general. One of the most succesful of 
these methods are Lindenmayer systems
introduced by Aristid Lindenmeyer

The AMAP (Atelje pour Modelisation de l'Architecture des Plants) approach.

Other methods to describe branching.

But the basic research always calls for applications. 

1.Individual trees

-radiation, photosynthesis

-mechanics

-hydraulic modelling

-wood quality

(e.g, visualize the quality of wood from forest to end products)

2.Forest

-dynamics of (mixed species) stands

-forest damage 

-matter exchange

3.Commercial Applications

-Entertainment industry

-Landscape Architecture

