\section{Getting Started with LIGNUM}

The  main design  principle in  \lignum is  to model  a tree  with few
simple  basic units that  correspond to  the organs  of the  tree. The
units  must  be  such that  they  allow  a  detailed analysis  of  the
structure and the development of the tree during its life time and and
still keeping  the model  computationally manageable to  create forest
stands consisting of different  tree species. Ideally the units should
be such that they can  be divided into more detailed tree compartments
if necessary.

Currently in  \lignum a tree is  modelled with four  units.  The three
basic units are  tree segment, branching point and  bud and the fourth
is an aggregate of these basic units called axis. 

The tree segment consists of  sapwood, heartwood, bark and foliage and
it captures the main metabolic  functiong in a tree.  The tree segment
corresponds roughly  to the annual  shoot but in the  strict modelling
sense  this  is  not  necessarily  always the  case,  especially  when
modelling heartwood trees.  The branching point is the place where two
or more  tree segments  are connected  to each other.   The bud  is an
embryonic shoot, the growing point  of the tree from which new shoots,
leaves  and flowers  may develop.   Buds can  be further  divided into
terminal buds located  at the top of axises  and lateral (or axillary)
buds located in the tree  segment.  More precisely for heartwood trees
lateral buds are located where the  leaf petiole is attached to a tree
segment. The aggregate unit axis corresponds  to a stem or a branch of
a tree.

The  design of  any  computer  program includes  the  decision how  to
represent information  and concepts from the real  world. C++ supports
many   programming   methodologies   (perhaps  most   notably   object
orientation)  but  the  emergence  of the  Standard  Template  Library
emphasizes   generic   programming   with  abstract   datatypes   that
encapsulates data and algorithms  working with this data. Two language
constructs  support  this approach:  classes  and  templates. A  class
defines a datatype consisting a  set of possible states and operations
defining transitions between those states. Template is simply C++ term
for a parameterized type. 
