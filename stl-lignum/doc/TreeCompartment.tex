\section{TreeCompartment}

The class TreeCompartment  is  the  common  base  class for  all  tree
compartments.  It defines  the position and  the direction  of a  tree
compartment but the  main purpose of the class  is to define  a common
set   of metabolic  processes   implemented as   virtual  methods. 

The  metabolic processes can  be further  divided into two categories.
The first category assumes  that no flow of  information from one tree
compartment  to another  is  required (thus also  not  allowed).  This
includes phosynthesis, respiration and mortality of tree compartments.
The   second   category  allows  flow  of   information  between  tree
compartments.   These   metabolic   processes include  length  growth,
diameter growth and flow of nutrients, water etc. in the tree.

The metabolic processes not   allowing parameter passing  assumes that
all the information must be present in the tree compartment before the
method is applied.  Computationally  these processes can be thought to
be parallel.   To apply these  throughout  the whole tree  use ForEach
algorithm using  a   predefined  functor simply calling    appropriate
method.   The  metabolic processes allowing  parameter  passing can be
applied throughout the  whole tree by  using either  AccumulateDown or
PropagateUp algorithms depending on  the direction of  the information
flow in the tree.   The actual type  of the parameter implementing the
flow of information between tree compartments is left to be decided in
the user defined functor used in AccumulateDown or in PropagateUp.

The important design  issue is weather  the metabolic processes should
be pure virtual or not.  Conceptually the better  solution would be to
make them pure virtual.  The class TreeCompartment is  meant to be  an
abstract base  class with  no  instances. But this  could (and  would)
cause practical  problems by requiring   to implement  each method  at
least as an empty method in each  subclass.  This is why the metabolic
processes are predefined as empty in this base class.
  
\subsection{The Class Declaration}
\begin{verbatim}
template <class TS,class BUD> 
class TreeCompartment{
  friend Point<METER> GetPoint(const TreeCompartment<TS,BUD>& tc);
  friend PositionVector GetDirection(const TreeCompartment<TS,BUD>&
tc);
  friend Tree<TS,BUD>& GetTree(const TreeCompartment<TS,BUD>& tc);
public:
  TreeCompartment();
  TreeCompartment(const Point<METER>& p, const PositionVector& d, Tree<TS,BUD>* t);
  virtual ~TreeCompartment();
  virtual void photosynthesis(){}
  virtual void respiration(){}
  virtual void mortality(){}
  template <class LGInOut> virtual void lengthGrowth(LGInOut& data){}
  template <class DGInOut> virtual void diameterGrowth(DGInOut& data){}
  template <class UpInOut> virtual void upFlow(UpInOut& data){}
  template <class DownInOut> virtual void downFlow(DownInOut& data){}
protected:
  Point<METER> point;
  PositionVector direction;
  Tree<TS,BUD>* tree;
};
\end{verbatim}

\subsection{Functions an operations}
\begin{enumerate}
\item \bf GetPoint \rm Returns the position of the tree compartment.
\begin{description}
    \item [Signature] friend Point<METER> GetPoint(const
    TreeCompartment<TS,BUD>\& tc).
\end{description}
\item \bf GetDirection \rm Returns the direction of the tree compartment.
\begin{description}
    \item [Signature] friend PositionVector GetDirection(const TreeCompartment<TS,BUD>\& tc).
\end{description}
\item \bf GetTree \rm Returns the tree the tree compartment
belongs to.
\begin{description}
    \item [Signature] friend Tree<TS,BUD>\& GetTree(const TreeCompartment<TS,BUD>\& tc).
\end{description}
\end{enumerate}

\subsection{Constructors}

The default constructor predefines the position at (0,0,0) and
the  direction  to be  up,   i.e.,  (0,0,1). 
\begin{description}
    \item [Signature 1] TreeCompartment().
\end{description}

The second constructor normalizes the direction to avoid subtile bugs
when visualizing the tree with OpenGL graphics library.
\begin{description}
    \item [Signature 2] TreeCompartment(const Point<METER>\& p, const
    PositionVector\& d, Tree<TS,BUD>* t)).
    \begin{description}
        \item [Arguments] for the constructor.
        \begin{description}
            \item [p] The position of the tree compartment.
            \item [d] The direction of the tree compartment . Will be
    normalized.
            \item [t] The tree the tree compartment is part of.
         \end{description}
    \end{description} 
\end{description}

\subsection{Destructor}

The destructor is predefined empty but allows the use of polymorphism
in the program.

\subsection{Metabolic processes}

The metabolic processes are predefined as empty methods but they designate a common
interface that allows the use of polymorphism in the program.
\begin{enumerate}
\item Photosynthesis
  \begin{description}
     \item [Signature] virtual void photosynthesis().
  \end{description}
\item Respiration
  \begin{description}
     \item [Signature] virtual void respiration().
  \end{description}
\item Mortality
   \begin{description}
     \item [Signature] virtual void mortality().
   \end{description}
\item Length Growth
   \begin{description}
       \item [Signature] virtual void lengthGrowth(LGInOut\& data).
       \begin{description}
           \item [Arguments] to the method.
           \begin{description} 
                \item [data] The data needed for the length growth,
     e.g., the $\lambda$. The type (LGInOut)of the argument is parameterized 
     and must be defined in the functor used in the appropriate generic function
     PropagateUp, AccumulateDown etc.
            \end{description}
       \end{description}
   \end{description}
\item Diameter Growth  
 \begin{description} 
     \item [Signature]  virtual void diameterGrowth(DGInOut\& data).
     \begin{description}
         \item [Arguments] to the method.
            \begin{description} 
                \item [data]  The data needed for the diameter
     growth. The type (DGInOut) is parameterized.
             \end{description}
     \end{description}
 \end{description} 
\item Up Flow  
  \begin{description} 
     \item [Signature] virtual void upFlow(UpInOut\& data).
     \begin{description}
         \item [Arguments] to the method.
            \begin{description} 
                \item [data]  The data needed for the up flow
     The type (UpInOut) is parameterized.
             \end{description}
     \end{description}
  \end{description}
\item Down Flow  
  \begin{description} 
     \item [Signature] virtual void downFlow(DownInOut\& data).
     \begin{description}
         \item [Arguments] to the method.
            \begin{description} 
                \item [data]  The data needed for the down flow
     The type (DownInOut) is parameterized.
             \end{description}
     \end{description}
  \end{description}
\end{enumerate}

\subsection{Data Members}
\begin{enumerate}
\item \bf Point<METER> p \rm The position of the tree compartment.
\item \bf PositionVector d \rm The direction of the tree compartment.
\item \bf Tree<TS,BUD>* t \rm The tree tree compartment belongs to.
\end{enumerate}