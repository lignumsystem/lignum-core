\section{Programming Style}

Lignum has evolved from  a small research experiment  to a medium size
project with many international connections.   Thus in the future also
the programming work to develop the model for diverse purposes will be
done by several  independent teams of programmers \footnote{Motto:  Do
not ask what Lignum can do for you; ask  what you can do for Lignum.}.
Hopefully the result of  the programming work done  in one team can be
shared by  other teams.  A  few simple  guidelines for the programming
work are presented to aid in this.  Please adapt  them so it is easier
for you to read and understand programs  written by others (and others
to  read the  programs  written  by  you).  First,   a   few lines  of
fictitious C++ code provides an example of the things to consider.

\begin{verbatim}
//A sample header (.h) for a class
typedef int METER;
typedef double KG;

class AClass{
friend AFriendFunction(METER first_argument, double second_argument);
public:
   AClass();
   METER firstMethodForTheClass();
private:
   KG Wf;
   double rho;
};

//A sample source (.cc) file for class methods
//The purpose of the method is to...
METER AClass::firstMethodForTheClass()
{
}
\end{verbatim}

There are many  suffixes to  denote C++  source and  header files. The
most common are probably .cc and .h. But the suffixes  you are used to
will do.

The C++ class names are capitalized.  If  the class name is a compound
word each word is capitalized. The method names  in a class start with
a  lower letter.  If the method   name is a    compound word the  each
following word is capitalized. The class attributes names are in lower
case. If the class  attribute name is a  compound  word the words  are
separated with the under score letter.

To choose a name for  a class attribute try  to be consistent with the
paper published or to be published.  For example  in a tree segment we
need to  have a state variable  for foliage mass.  In publications the
symbol for the foliage mass is $W_f$.  So the name of the attribute in
the class TreeSegment should be \tt Wf\rm.  If a  Greek letter is used
in a paper as a state variable or a parameter write it out as the name
of the class attribute.  For example the symbol  $\rho$ is used as the
wood density.  So the name of the attribute  should be \tt rho\rm.  In
general we use \LaTeX \ style to write out the mathematical symbols.

All   functions are   written as class   names,  i.e.,   each word  is
capitalized.   The    arguments to  functions  are  written   as class
attributes.  Try to  avoid macros.  Use  inline functions instead. All
user  defined types (so   called typedef declarations) are written  in
capital letters.

The indentation of  a program code is  also important. Lignum has been
developed in an  UNIX system so the  proper indentation is the one you
get simply by using emacs.  In fact emacs  is a powerful editor and it
provides   you with features  like     automatic coloring of   program
components, listing  of  methods and  functions in  a source file etc.
But if you have used other programming  environment (in a PC) and want
to port the  code  to Lignum  we  don't  require you to   reindent the
code. However  you can do  this  easily by  using emacs  function that
(re)indents the whole file according to the programming language used.

And  finally  a  few word  about comments.   Needles  to say  they are
important and you  can also use  them when creating the documentation.
But  remember comments are \it  not \rm a documentation. To understand
the  use  of a class,   a method  or  a  function examples  are always
valuable.

Don't  bother to explain trivial pieces  of program (like ``this is an
assignment'',  ``this statement increases the  loop variable by one'')
but try to answer questions like \it why \rm and \it how \rm I'm doing
a method or  a  function or  a tricky  algorithm in  it, what  are the
preconditions and postconditions of a method (what must be true before
the  method is called and  what is true after  the method has finished
respectively).  

Naturally you  can place the comments where  ever they are  needed but
describe the  purpose of the method or  a function in a comment placed
just  before the beginning of  the method  or function.  Remember that
what may seem obvious for you is not necessarily clear for others.

These  few guidelines should  be enough.  Note   that these are merely
textual issues that aid  in reading and understanding programs written
by others. Good programming practices are presented in C/C++ textbooks
and especially good  programming methodology suitable  for C++  can be
learned by studying Standard Template Library (STL).

Additional requirements will be explained in this section if they turn
out to be necessary.
