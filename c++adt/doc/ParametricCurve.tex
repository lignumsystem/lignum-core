
\section{ParametricCurve}

The  class  \tt ParametricCurve  \rm  can  be  used to  define  simple
functions.  A function $y = f(x)$ can  be described in an  ASCII file
that  contains a  set  of $(x,y)$  pairs  defining the  values of  the
function at given  points.  There is no limit to  the number of points
that can be defined. At least two points must be defined.

The $x$ values must be in ascending order and each $x$ value must have
exactly one $y$ value.  Functions  having steps can be approximated by
having two close enough $x$ values.  The closer the two $x$ values are
the better the step in the function is described.

The values of the function  between the defined pairs are interpolated
with  a   straight  line.  Values   outside  the  defined   pairs  are
extrapolated  with  a straight  line  defined  by  first two  or  last
two $(x,y)$ pairs found in the definition file.

\subsection{Using Parametric Curve}
 
An example of a file and the parametric curve it defines 
is in Figure~\ref{fig:curve}. Unfortunately the current implementation
does not allow the user to include any comments in the file. 

Assuming the name of the file is \tt curve.fn \rm the following 
sequence of operations reads the file and evaluates the value of the
curve at $x = 2.0$.
\begin{tabbing}
Tabbing \= Tabbing \= Tabbing \= Tab \= \kill
\>\>\>\tt  ParametricCurve pc(``curve.fn''); \\
\>\>\>\tt  if (pc.ok()) \\
\>\>\>\>\tt pc(2.0);
\end{tabbing}
First, the parametric curve is created and then after checking
if the program was able to read the file the value of the curve
is evaluated. The classes with overloaded function operators are
also called functors.

\begin{figure}[h]
\begin{center}
\begin{picture}(330,250)
%\graphpaper(0,0)(330,250)
\put(0,0){
  \begin{picture}(50,250)
  \put(55,220) {File ``curve.fn''}

  \put(50,160){\begin{tabular}{r r} 
                    -7.0 & -3.0 \\ 
                    -1.0 &  0.0 \\ 
                     2.0 &  5.0 \\ 
                     8.0 &  -3.0 \\
                     10.0 &  3.0 \\
                     13.0 &  4.0 \\
  \end{tabular}
  }
  \end{picture}
}

\put(50,0){
  \begin{picture}(100,250)
    \put(100,220) {The Parametric Curve}

    \put(60,70){\line(2,1){60}}
    \put(120,100){\line(1,2){30}}
    \put(150,160){\line(2,-3){60}}
    \put(210,70){\line(1,3){20}}
    \put(230,130){\line(3,1){30}}

    \put(60,70) {\circle*{2}}
    \put(120,100){\circle*{2}}
    \put(150,160){\circle*{2}}
    \put(210,70){\circle*{2}}
    \put(230,130){\circle*{2}}
    \put(260,140){\circle*{2}}

    \put(20,100){\vector(1,0){250}}
    \put(130,10){\vector(0,1){200}}

    \put(60,60){$(-7.0,-3.0)$}
    \put(120,90){$(-1.0,0.0)$}
    \put(132,165){$(2.0,5.0)$}
    \put(210,60){$(8.0,-3.0)$}
    \put(190,135){$(10.0,3.0)$}
    \put(245,145){$(13.0,4.0)$}
  \end{picture}  
}
\end{picture}
\caption{Defining a Parametric Curve}\label{fig:curve}
\end{center}
\end{figure}

\subsection{The Class Declaration ParametricCurve}
\verbinput{../include/ParametricCurve.h}

\subsection{Public Methods}
\subsubsection{ParametricCurve}
The three overloaded constructors. The first (signature 1) does not define
any parametric curve. Use method \tt install \rm  to define
the curve. 
 
\begin{description}
    \item [Signature 1] ParametricCurve().
    \item [Signature 2] ParametricCurve(const vector<double>\& v).
    \item [Arguments] for the method.
	\begin{description}
           \item [v] The vector containing $(x,y)$ pairs for $y =
    f(x)$. 
        \end{description}
     \item [Signature 3] ParametricCurve(const string\& file).
    \item [Arguments] for the method.
       \begin{description}
           \item [file] The file containing $(x,y)$ pairs for $y =
    f(x)$. 
         \end{description} 
    \item [Returns] Nothing.
\end{description} 

\subsubsection{operator()}
Evaluate the value $y$ of the curve at the point $x$. 
See the method \tt eval\rm.
\begin{description}
   \item [Signature] double  operator() (double x)const.
   \item [Arguments] for the method.
     \begin{description}
        \item [x] The point $x$ where the value of the parametric curve 
                  is to be evaluated.
       \end{description}
    \item [Returns] The value $y$ at the point $x$.
\end{description} 

\subsubsection{install}
Install a parametric cure defined in the file \tt file\rm.
 \begin{description}
    \item [Signature] bool install(const string\& file).
    \item [Arguments] for the method
       \begin{description}
         \item[file] The file containing  $(x,y)$ pairs for $y =
    	f(x)$.
        \end{description} 
    \item [Returns] true if the parametric curve was properly defined,
        	               false if not (see method \tt ok\rm).
 \end{description} 


\subsubsection{ok} 
Checks if the parametric curve is properly defined. The curve is 
properly defined if each $x$ value has exactly one $y$ value in the file and
the $x$ values are strictly in ascending order.
   \begin{description}
    \item [Signature] bool ok().
    \item [Returns] true if  the parametric curve properly defined,
                       false if not. 
   \end{description} 

\subsubsection{eval}
Evalute the value $y$  of the curve at the point $x$.
\begin{description}  
    \item [Signature] double eval(double x).
    \item [Arguments] for the method.
      \begin{description}
        \item [x] The point $x$ where the value of the parametric curve 
                  is to be evaluated.
       \end{description}
    \item [Returns] The value $y$ at the point $x$.
\end{description} 

\subsubsection{getFile}
\begin{description}
    \item [Signature] string  getFile().
    \item [Returns] The file where the curve is defined.
\end{description}

\subsection{Private Methods}

\subsubsection{read\_xy\_file}
Read the file and create the internal representation of the parametric curve.
 \begin{description}
     \item [Signature] ParametricCurve\& read\_xy\_file(char*
 file\_name).
    \item [Arguments] for the method.
      \begin{description}
        \item [file] The file where the parametric curve is defined.
       \end{description}
    \item [Returns] The parametric curve itself.
 \end{description} 

\subsection{Private Data Members}
\begin{description}
 \item [file] The name of the file where the parametric curve
                      is defined.
 \item [v] Vector containg the internal representation of the
                          parametric curve. 
 \item [num\_of\_elements]  The length of the vector \tt v\rm.
\end{description}
