%\addcontentsline{toc}{section}{Introduction}
\section{Introduction}

This document describess useful,  general purpose classes designed and
implemented  during   the  development  process  of   the  tree  model
\lignum.  The classes  described here  are  not found  in the  current
implementations  of the C++  Standard Template  Library (STL)  but are
frequently used during our work.  By saying that the classes described
in this report  are ``general purpose'' it means  that the classes are
not specific to the functioning of the tree model \lignum\ itself, but
can be used in many other programs as well.

The implementations of the classes are collected into a library called
\linebreak  libc++adt.a  that   provides  basic  algorithms  and  data
structures  needed.  The  STL  provides many  abstract  datatypes  for
generic programming, so if you  want to contribute to libc++adt.a, you
are welcome  to do  so, but  check first if  STL already  provides the
abstract datatype you need.

Many classes in libc++adt.a use  templates and are built using classes
in STL  so STL  should also be  available. Naturally, you  should make
sure that  you understand  how templates work  in C++  (especially the
instantiation mechanism of your compiler) before using this library.

To use a class described in this report include the header file of the
class into your program and  link the program against libc++adt.a. The
name of  the header file for a  class is the class  name appended with
.h.  For example, to use PositionVector include PositionVector.h.
