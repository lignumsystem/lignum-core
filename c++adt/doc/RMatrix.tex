\section{RMatrix}

The class \tt RMatrix \rm can be used to define rotations in
a right handed coordinate system. The rotation  matrices
are abstracted from the user. The user defines the direction
(axis) and the angle of the rotation (in radians) and the 
overloaded function operator returns the corresponding 
transition matrix. This transition matrix can then be applied 
to a vector (\tt vector \rm type available in STL) representing a point
in three dimensional space. See also \tt PositionVector \rm
that implements a position vector in a three dimensional coordinate 
system.

\subsection{Rotation Matrices}\label{ssc:rm}

Each of the rotations about $x$, $y$ and $z$-axes can be implemented with
the help of corresponding $3 \times 3$ rotation matrices. 
The rotation $R_{z}(\alpha)$ about $z$-axes by angle 
$\alpha$ can be represented 
by the rotation matrix:

\begin{displaymath}
R_{z}(\alpha) =
\left[ 
\begin{array}{ccc}
\cos\alpha & \sin\alpha & 0 \\
-\sin\alpha & \cos\alpha & 0 \\
0 & 0 & 1 
\end{array}
\right]
\end{displaymath}  
 
The rotation $R_{x}(\alpha)$  about x-axes by angle $\alpha$ 
can be represented by rotation matrix:

\begin{displaymath}
R_{x}(\alpha) =
\left[ 
\begin{array}{ccc}
1 & 0 & 0 \\
0 & \cos\alpha & -\sin\alpha  \\
0 &\sin\alpha  &  \cos\alpha
\end{array}
\right]
\end{displaymath}  

Finally, the rotation $R_{y}(\alpha)$  about y-axes by angle $\alpha$
can be represented by rotation matrix:

\begin{displaymath}
R_{y}(\alpha) =
\left[ 
\begin{array}{ccc}
\cos\alpha & 0 & -\sin\alpha \\
0 & 1 & 0 \\
\sin\alpha & 0 & \cos\alpha
\end{array}
\right]
\end{displaymath}  

See any text book in linear algebra for further details.

\subsection{Using RMatrix}

The \tt RMatrix \rm abstracts the rotation matrices and their
matrix operations from the user. To compute a rotation
in a three dimensional space the user first specifies
the axis the rotation will be about, in this case $x$-axis:

\begin{tabbing}
Tabbing \= Tabbing \= Tabbing \= \kill
\>\>\> \tt RMatrix r(ROTATE\_X);
\end{tabbing}

The effect of this operation in practice is that the right rotation matrix 
is chosen, in this case $R_{x}(\alpha)$ in section~\ref{ssc:rm}.
To define rotation about $y$ and $z$-axis user can write 
\tt ROTATE\_Y \rm and \tt ROTATE\_Z \rm respectively. 

The next step is to define the angle of rotation and create the
corresponding transfer matrix. In this example the angle of rotation
is \tt pi \rm and it is assumed to be the symbolic value of $\pi$:

\begin{tabbing}
\tt
Tabbing \= Tabbing \= Tabbing \= \kill
\>\>\>\tt TMatrix<double> m(3,3); \\
\>\>\>\tt  m = r(pi);
\end{tabbing}

Finally, a point (given as a vector) can be transferred 
to a new position by a simple multiplication:

\begin{tabbing}
Tabbing \= Tabbing \= Tabbing \= \kill
\>\>\>\tt vector<double> v(3); \\
\>\>\>\tt v[0] = 1.0; v[1] = 2.0; v[2] = 3.0; \\
\>\>\>\tt v = v * m;
\end{tabbing}

The vector \tt v \rm should now represent the point (?,?,?).

The class \tt PositionVector \rm abstracts the \tt RMatrix \rm
from the user. You may find it more suitable for
your needs.

\subsection{The Class Declaration of RMatrix}
\begin{verbatim}
class RMatrix{
  friend RADIAN fn_1(RADIAN angle);
  friend RADIAN fn_0(RADIAN angle);
  friend RADIAN neg_sin(RADIAN angle);
  friend RADIAN neg_cos(RADIAN angle);
public:
  RMatrix(ROTATION direction);
  TMatrix<double> operator()(RADIAN angle)const;
  RMatrix& changeAxis(ROTATION direction);
  RMatrix& transpose();
  RMatrix& inverse();
private:
  ROTATION r_direction;
  RADIAN (*(fn_matrix_table[3][3*3]))(RADIAN);
};
\end{verbatim}

\subsection{Public Methods}
\subsubsection{RMatrix(ROTATION direction);}
The constructor for the class. The argument specifies the axis 
the rotation will be about.
 \begin{description}
    \item [Arguments] for the method.
      \begin{description}
        \item [direction] The axis of rotation. One of \\
                       \tt ROTATE\_X\rm, \tt ROTATE\_Y \tt or 
                       \tt ROTATE\_Z\rm.
      \end{description}
    \item [Returns] Nothing.
 \end{description}
\subsubsection{TMatrix<double> operator()(RADIAN angle)const;}
The overloaded function operator implementing the rotation.
 \begin{description}
    \item [Arguments] for the method.
      \begin{description}
         \item [angle] The angle of rotation in radians.
      \end{description}
    \item [Returns] The transition matrix.
 \end{description}
\subsubsection{RMatrix\& changeAxis(ROTATION direction);}
Changes the axis of rotation.

 \begin{description}
    \item [Arguments] for the method.
      \begin{description}
        \item [direction] The new axis of rotation. Possible values are
                          one of \tt ROTATE\_X\rm, \tt ROTATE\_Y \rm
                          or \tt ROTATE\_Z\rm. 
      \end{description}
     \item [Returns] The rotation matrix itself.
  \end{description}
\subsubsection{RMatrix\& transpose();}
Implements the transpose of the current rotation matrix. Can be used to undo
the immediate previous rotation (Naturally, the rotation can also be undone
by negating the angle).
     
    \begin{description}
       \item [Returns] The rotation matrix itself.
    \end{description} 

\subsubsection{RMatrix\& inverse();}
Implements the inverse of the current rotation matrix. The inverse
of a rotation matrix is its transpose. See any text book in linear algebra 
for discussion about orthonormal bases.

    \begin{description}
       \item [Returns] The rotation matrix itself.
    \end{description} 

\subsection{Functions and Operations on RMatrix}
There are four simple functions that needed to implement rotation
matrices. \tt fn\_1 \rm returns 1.0 and \tt fn\_0 \rm returns 0.0. 
\tt neg\_sin \rm returns $-\sin$ and \tt neg\_cos \rm returns $-\cos$. 
These are not meant to be used directly by the user of the class 
but are used to construct the rotation matrices 
$R_{x}(\alpha)$, $R_{y}(\alpha)$ and $R_{z}(\alpha)$ (Section 
\ref{ssc:rm}) in \tt fn\_matrix\_table\rm.
 
\subsection{Private Data Members}
    \begin{description}
       \item [ROTATION r\_direction] Index to the row of \tt
       fn\_matrix\_table\rm,  the matrix of rotation matrices 
       thus defining the axis of current rotation. Possible
       values are one of \tt ROTATE\_X\rm, \tt ROTATE\_Y \rm
       or \tt ROTATE\_Z\rm. 
    
       \item [fn\_matrix\_table] The $3 \times 9$ matrix holding rotation 
       matrices.  There are three rotation matrices and each matrix is
       $3 \times 3 = 9$. The elements in the matrices are
       functions of type double taking one argument of type
       double (e.g., \tt double sin(double) \rm defined in the
       C-library). 
       So from the point of view of the C++ compiler
       the type of \tt fn\_matrix\_table \rm is a vector
       of functions of type double taking one argument of type
       double. 

 
                   


    \end{description} 



